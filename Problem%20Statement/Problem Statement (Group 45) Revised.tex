\documentclass[journal, 11pt, letterpaper]{IEEEtran}
\usepackage{blindtext}
\usepackage{graphicx}
\newcommand\tab[1][1cm]{\hspace*{#1}}
\onecolumn
\rmfamily
\usepackage[margin=0.75in]{geometry}


%****************BEGINNING OF TITLE PAGE****************


\title{Problem Statement for The Boeing Research Project{}}
\author{Brandon To, Mike Truong, Michael Chan}


\begin{document}
\clearpage\maketitle
\thispagestyle{empty}


\begin{abstract}
\tab A focused workforce is able to accomplish multiple task within the time allocated. The direction of this project is to define a plan that determines the most efficient way for salaried Boeing employees to get to work on time to minimize the disruption to both their production and personal lives. This project will be experimental using the technique of a case study to examine Boeing employees over a time duration of two work days. 
Our plan is to use data collection techniques and analytics to develop an algorithm that can be used as demands change. 
The data collection involves evaluating instances that can be improved within employee work schedules with the use of surveys, questionnaires, observation, and records. 
Combining faculty documents with the data obtained from the employees of Boeing Everett the findings will determine a method to improve efficiency. At length the results will boost productivity for Boeing Everett. 
\end{abstract}
%****************END OF TITLE PAGE****************


%****************TABLE OF CONTENTS*******************
\newpage
\clearpage
\thispagestyle{empty}
\tableofcontents
\newpage
%****************END OF TABLE OF CONTENTS**************


%****************BEGINNING DOCUMENT****************
\setcounter{page}{1}




\section{Problem Definition}
\tab The Boeing Company has addressed the demand for optimizing their scheduling of salaried employees to reduce disruptions in terms of commuting practices and start/stop times. At the Boeing Everett Factory in particular, inefficient arrivals and departure times cause a direct correlation to the employee’s ability to balance production efficiency and time spent with their families.
Taking into consideration of the large population of approximately 30,000 employees and unorganized shift times, there are many instances where a dense amount of workers would attempt to department the Everett Facility at the same time, bringing about unnecessary traffic congestion.
It is imperative that large factories like Boeing remain as efficient to not only ensure that the optimal amount of progress is made at the work environment, but also, ensuring that their employees are content and commute safely to their homes.
\tab This project’s main focus will be placed on salaried employees rather than those on hourly wages. 
There is currently an unknown amount of start times for salaried employees at Boeing which include engineers, management, and analysts. 


\section{Proposed Solution}


\tab As a team, we will be conducting a detailed research to analyze the existing property at Everett, the demographics and methods of transportations of salaried employees, and compiling the thoughts and opinions of employees through a case study.
To accomplish this, team members will be required to analyze multiple drawings of the Everett Factory, involving maps of the site and parking structures. Team members will also be using data collection techniques including but not all, surveys, questionnaires, observation, and records.  
This is crucial as it allows for the allocation of salaried employees to enter and depart with the shortest distance in mind.
Team members will also be required to assess the demographics and transportation options of employees provided by the Boeing Everett factory in order to understand the average distances traveled and routes as they commute from home to work.
Lastly, the team will be required to travel to the Boeing Everett site to interview employees, gather data, and access the property for two days. 
During these two days, we will examine parking structures, transit routes, interviewing, observing and the overall building's layout at Boeing Everett. 
After the entire data has been compiled, we will be developing an algorithm that can be used to optimize the salaried employee’s arrival and departure times.
The algorithm shall be designed to consider a multitude of factors enabling to be flexible depending on demand requested by Boeing.




\section{Performance Metrics}
\tab The proposed solution will be tested on a select group or program within the Boeing Everett site to simulate its effectiveness prior to being fully active to the site as a whole. 
While the test population will encompass approximately a population of 30 or so employees depending on the decision of management, the proposal is intended to be used on approximately 20,000 employees with the aim to balance production efficiency and reduce disruptions within personal lives. 
The proposal will be determined as effective if there is a positive impact in efficiency compared with current statistics provided. 
In the event that the proposal is deemed successful with increasing productivity within the site, the standardization of the proposal is dependant on the site leadership. 
Allowing the proposal to be in effect and used as the site leaders deem fit, a survey will be conducted among the employees at Boeing Everett for feedback. 
Using this feedback the team will re-evaluated the proposal to provide both comfort and efficiency for Boeing Everett.    


%****************SIGNATURES****************
\vspace{10mm} %10mm vertical space
\section{Signatures}  %Replace the subsection part ASAP. It is only used for the format and is incorrect here.
\vspace{5mm} %5mm vertical space
\noindent
 \makebox[2in]{\hrulefill}\\\\
 \makebox[2in]{\hrulefill}\\\\
 \makebox[2in]{\hrulefill}\\\\
 \makebox[2in]{\hrulefill}\\\\


%****************END OF DOCUMENT****************


\end{document}